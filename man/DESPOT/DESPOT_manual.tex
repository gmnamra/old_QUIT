\documentclass{report}
\usepackage{amsmath}
\usepackage{authblk}
\usepackage[a4paper,vmargin={25mm,25mm},hmargin={20mm,20mm}]{geometry}
\usepackage{graphicx}
\usepackage[utf8]{inputenc}
\usepackage{lmodern}
\usepackage{setspace}
\usepackage{listings}
\lstset{basicstyle=\ttfamily\footnotesize,breaklines=true}
\usepackage[backend=bibtex,sorting=none]{biblatex}
\addbibresource{/Users/Tobias/Documents/Kings/Bibliography/rat_bib}

\begin{document}

\title{Manual for DESPOT Tools}
\author{Tobias C Wood}
\affil{King's College London, Institute of Psychiatry, Department of Neuroimaging}
\maketitle

\tableofcontents

\chapter{Introduction}

This is a manual for the several programs I have written to process Driven Equilibrium Single-Pulse Observation of T1\&2 (DESPOT) images, and the various extensions to these techniques which were mainly invented by Sean Deoni. Most of you reading this will be interested in multi-component DESPOT (mcDESPOT).

The subsequent sections mostly assume that you are processing a full mcDESPOT data set. If you are only interested in DESPOT1 or DESPOT2, then there are parts of the processing you can skip.

These programs have been written over the course of 18 months and have been revised multiple times as I have learnt more about DESPOT, the nifti file format and have re-learned some rusty C++ skills. Processing mcDESPOT data in particular is a difficult problem that I do not consider fully solved. I would estimate that these programs are the result of over 500 hours of work, most of which went into the \texttt{mcdespot} program.

\section{Credit / Methods}\label{credit}

I cannot write a general purpose methods section because most projects use subtly different acquisition protocols. However, you should include something along the lines of "images were processed using programs written in C++ using the Eigen matrix maths library~\cite{eigenweb}".

As of writing, I do not currently have a methods paper for people to cite. I am hoping this will change soon. In the meantime I am usually happy to be listed as an author on any papers using these programs (somewhere in the middle is fine). You should cite these papers by Sean Deoni: DESPOT1~\cite{Deoni:2007}, DESPOT2~\cite{Deoni:2009a}, mcDESPOT~\cite{Deoni:2008c, Deoni:2012}.

\printbibliography[heading=subbibliography]

\section{Accessing These Programs}

If you are a member of the Institute of Psychiatry, all of these programs are available to use on the Neuro-Analysis Network (NAN). They are installed in my public area, \texttt{/software/local/k1078535/}. Inside that directory is a script called \texttt{set\_paths.sh}. You need to \texttt{source} this script in order to use the programs. Preferably you should add the line \texttt{source /software/local/k1078535/set\_paths.sh} to the end of your \texttt{.cshrc} file in your home area.

If you are not a member of the Institute of Psychiatry, I hope I have already given you the programs to run, otherwise I'm not sure how you got hold of this document. At some point they will be available to download from the King's College website.

\chapter{Preprocessing}

\section{Data Required}

In general, DESPOT requires the following of scans.
\begin{enumerate} \itemsep1pt \parskip0pt \parsep0pt
  \item A Spoiled Gradient Recalled (SPGR) image at multiple flip-angles
  \item A balanced Steady-State Free Precession (bSSFP, FIESTA or true FISP) image at multiple flip-angles and at least 2 phase-cycling patterns.
  \item A B1 (RF inhomogeneity) map.
\end{enumerate}

For most data acquired at the IoP, the B1 map will be in the form of an Inversion Recovery SPGR (IR-SPGR) scan, the SPGR and SSFP scans will have 8 flip angles and there will be two phase-cycling patterns (180 and 0 degrees).

\section{File Formats}

All my programs use the NIFTI file format (\texttt{.nii}). They will read zipped files (\texttt{.nii.gz}). I do not provide help with conversion from whatever file format your scanner uses (likely DICOM), because there is no way I can keep track of the subtly different acquisition protocols that have been used.

If you use the GE scanners at the IoP then it is likely that the two phase-cycling patterns for SSFP will be contained in the same file. It is preferred to keep them like that instead of splitting them into two separate files.

\section{Registration and Resampling}

As with most MRI techniques, the effects of sample motion must be minimised before processing. This can be achieved by registering all the images together using either FSL, SPM, or your software package of choice. Because the contrast of the SPGR and SSFP images changes with the flip-angle, mutual information should be used as the registration cost function.

The scans must also be resampled to the same matrix size, as often the B1 map will be acquired at lower resolution. As scanners (particularly the GE scanner) often zero-fill images before reconstruction, it is HIGHLY recommended that this resampling is to the original matrix size used for the SPGR and SSFP images. For current protocols this is usually around 128x128x96. Processing at a larger matrix size will not improve results but will seriously lengthen processing time.

A template script (\texttt{template\_preprocess.sh}) that will carry out all these steps has been written by Anna Combes and should have been distributed with this manual. It assumes that all your converted NIFTIs are placed in one directory. It uses FSL to split the files into individual images for each flip-angle, register them together, merge them back for processing, and also creates a brain mask for you.

You will need to edit the filenames on lines 14, 15 and 24 to match the filenames you used when converting your data. You may wish to add commands to delete the \texttt{splitfiles} directory it produces to save space.

You will also need to edit line 16 to provide a target image to register to, with  a sensible matrix size as described above. When the template was written our files had been zero-filled to 256x256x192, so simply subsampling the first SPGR image by 2 created a target with the correct matrix size. You may wish to register to a template, or create a specific matrix size using \texttt{fslcreatehd}.

\chapter{Processing}

\section{Processing steps and programs}

Currently the following programs are available for use:
\begin{itemize} \itemsep1pt \parskip0pt \parsep0pt
  \item \texttt{afi}
  \item \texttt{despot1}
  \item \texttt{despot1hifi}
  \item \texttt{phasemap}
  \item \texttt{despot2}
  \item \texttt{despot2fm}
  \item \texttt{mcdespot}
\end{itemize}

All of these programs will print a usage message if you run them with the \texttt{-h} option (and no others). This message will list the various other options you can specify to each program. I have tried to be consistent with these options across all the programs. If you think the options could be improved please let me know. Some of the most useful options available are:

\begin{itemize} \itemsep1pt \parskip0pt \parsep0pt
	\item \texttt{-v} "Verbose" mode, prints extra information such as which slice is being processed.
	\item \texttt{-m maskfile.nii.gz} Only process voxels which are non-zero in the specified file.
	\item \texttt{-s M -p N} Start processing at slice M, stop at N. Only available in \texttt{despot2fm} and \texttt{mcdespot}.
	\item \texttt{-o prefix} Add a prefix to the start of all output filenames.
\end{itemize}

With the exception of \texttt{afi} these programs also require sequence information such as the flip angles and TR used. The programs will prompt you to enter these when needed. See section \ref{scripting} for a guide on scripting the programs so you do not have to type this information in repeatedly.

Depending on your project and data you will only need some of the above programs. If you want to process absolutely everything the steps required are:

\begin{enumerate}\itemsep1pt \parskip0pt \parsep0pt
	\item Create a B1 map (use \texttt{despot1hifi} or \texttt{afi})
	\item Create a T1 map (use \texttt{despot1})
	\item Create a T2 map (use \texttt{despot2fm})
	\item Create a Myelin Water Fraction (MWF) map (use \texttt{mcdespot})
\end{enumerate}

If you are only interested in mcDESPOT you can skip steps 1 and 2.

The output of each program is prefixed by its name, e.g. the output from \texttt{despot1hifi} is three files called \texttt{HIFI\_T1.nii.gz}, \texttt{HIFI\_PD.nii.gz} and \texttt{HIFI\_B1.nii.gz}. The output of the \texttt{mcdespot} program is prefixed by the number of components selected to model.

\section{A Note on Units}

All times, e.g. TR, TI and the output T1 and T2 maps, are measured in SECONDS by all the programs. They are NOT measured in milliseconds. Make sure you enter them correctly. If you want to convert the output to milliseconds then use \texttt{fslmaths} and multiply by 1000.

Off-resonance (f0) maps are produced in Hertz (Hz). B1 maps are fractions of the desired flip angle. Myelin Water Fraction and Cerebral-Spinal Fluid Fractions are fractions, obviously. Proton Density is in arbitrary units and will depend on your scanner's scaling factors.

\section{A Note for Agilent Scanners}

I have automated steps in the processing for files coming from an Agilent scanner. If you have used my Agilent image conversion tools you will not have to enter flip angles, TRs, etc. by hand, they will automatically be read from the nifti header.

\section{B0 Maps}

No IoP studies that I am aware of have collected magnetic field (B0) inhomogeneity maps. I have only included this section to mention that a program to help with this, \texttt{phasemap}, exists for future use.

\section{B1 Maps}

The first step in any DESPOT analysis is to create a map of the RF field inhomogeneity which is usually shortened to a B1 map. We need this because B1 inhomogeneity changes the flip angles that the scanner applies to different voxels in our image, and DESPOT relies on accurate knowledge of those flip angles. The map is measured as the ratio of the actual flip angle applied to the desired flip angle.

There are multiple B1 mapping techniques in the literature. I prefer to use a technique called Actual Flip angle Imaging (AFI), but most IoP data is acquired using Sean Deoni's DESPOT1-HIFI sequence.

\subsection{AFI}

Processing an AFI sequence is simple. The sequence uses interleaved TRs to collect two volumes, and the actual flip angle in each voxel is then directly calculated from the ratio of those two volumes. It is hence very simple to run, just type \texttt{afi input\_file.nii.gz}.

\subsection{DESPOT1-HIFI}

When Sean Deoni first started work on the DESPOT techniques the B1 mapping literature was not as developed as today. He created his own technique which involves taking an extra IR-SPGR scan with the normal SPGR, and then optimising across both to find the best B1. To my knowledge no-one uses this method except for Sean, but all DESPOT studies within the IoP to date have used this method. The basic usage is:

\begin{lstlisting}[language=sh]
despot1hifi spgrfile.nii.gz irspgrfile.nii.gz
\end{lstlisting}

The program will then ask you to enter the flip angles and TR used for the SPGR sequence, and the flip angles, TR, TI (inversion time) and number of acquired slices for the IR-SPGR sequence. It is extremely important to get this last piece of information right - you need the number of slices or "spatial locations" that the scanner was told to acquire, not the number it reconstructed.

The reason for this is that the program must calculate the ``true'' TI time used by the scanner. The GE scanner segments an IR acquisition in a way that means the TI time shown by the console is not accurate. If your scan was NOT acquired on a 3T MR750 GE scanner then this calculation will be incorrect. The program also contain a calculation that is valid for a 1.5T GE scanner which you can select with the \texttt{-i} option. If you did not use either of these scanners, the same option will also let you specify a ``raw'' TI if you know how to calculate it yourself.

\texttt{despot1hifi} produces a B1 map, a T1 map and a Proton Density (PD) map. However I do not recommend using the T1 map it produces directly. This is because B1 map produced invariably contains a large amount of artifactual anatomy, which is not consistent with the B1 literature. It must be smoothed to remove the anatomy and the T1 map recalculated using the \texttt{despot1} program.

Sean's version of this program automatically included this smoothing and recalculation step using a fixed Gaussian kernel. I have separated this out, both so people can clearly see what the program does, and as \texttt{fslmaths} includes many smoothing options which are better than a plain Gaussian. My personal preference is for median filtering, for instance:

\begin{lstlisting}[language=sh]
fslmaths inputB1.nii.gz -kbox 7 -fmedian smoothedB1.nii.gz
\end{lstlisting}

\texttt{despot1hifi} is a fast program, it should complete in only a few seconds.

\section{DESPOT1}

Once you have a good B1 map you can proceed to analyse your SPGR data and produce a good T1 map. This is the simplest of the analysis programs to run. This example applies the B1 map and also a mask:

\begin{lstlisting}[language=sh]
despot1 --B1 B1map.nii.gz -m mask.nii.gz spgrfile.nii.gz
\end{lstlisting}

The program will then you to enter the flip angles and TR, and then create two output files, a T1 map and PD map. The T1 map is in seconds. \texttt{despot1} is a fast program, it should complete in only a few seconds.

\section{A note on Display Programs}

\texttt{despot1} is an unconstrained fitting technique. This means that in areas of low SNR, i.e. everything outside the head, it will fit nonsense values including infinity. Most viewers then can't pick a suitable range of display values (what radiographers call the window), and just display black. The values are there, you just need to tell the viewer to pick a decent display range. In \texttt{fslview}, you can enter the minimum and maximum display values above the image next to the "wheels" in the interface. Note that the image will not display until you have entered a number in both boxes. For a T1 map, 0 and 3 should work.

If you use a mask file then this issue should be avoided, or if you want you can use \texttt{fslmaths} to threshold the images (using \texttt{-thr} and \texttt{-uthr}) before display. I do not threshold inside the program because values of infinity inside the brain is a big clue that there's a bug in my program that I need to fix.

\section{DESPOT2}

A T2 map can be created from the SSFP data and the T1 map produced from \texttt{despot1}. The classic DESPOT2 algorithm requires only the 180 degree phase-cycling scan, and is contained in \texttt{despot2}. Unfortunately it suffers banding artefacts caused by magnetic field (B0) inhomogeneity, and hence is practically useless at field strengths of 3T and above. We will hence not discuss it further.

To mitigate the banding artefacts, Sean invented DESPOT2 with Full Modelling, or DESPOT2-FM (continuing the audio theme from DESPOT1-HIFI). This technique relies on taking two scans with different phase-cycling patterns (Sean chose 180 and 0 degrees). A global fitting method (Stochastic Region Contraction) is then used to find the value of T2 simultaneously with the amount of off-resonance (f0).

The program requires a T1 map. It can be run like this:

\begin{lstlisting}[language=sh]
despot2fm --B1 B1map.nii.gz -m mask.nii.gz T1map.nii.gz ssfpfile1.nii.gz ssfpfile2.nii.gz
\end{lstlisting}

and it will then produce a T2 and f0 map. The number of SSFP files you enter as input depends on how you have acquired your data, and how you wish to normalise the signals. I highly recommend a single file, organised as all the flip angles for the 180 degree phase-cycling followed all the flip angles for the 0 degree phase-cycling. The program will prompt you for the number of phase-cycling patterns in each file, the phase-cycling patterns, the flip-angles and the TR used.

To produce an accurate fit it is necessary to normalise the data on a per-voxel basis to the mean across flip-angles. If the phase-cycling patterns are contained in one file this will be done across phase-cycling patterns as well. This improves the results in the banding artefact regions. If you do not wish to use this normalisation step you can change the scaling mode with the \texttt{-s} option. The program will then attempt to fit a PD map as well.

If your data did not come from a 3T scanner, then you will need to change the field strength as well with the \texttt{-t} option. This will be covered further when discussing \texttt{mcdespot}.

\texttt{despot2fm} is a moderately slow program. It should take around 30 minutes to process a whole brain. If you do not mask the image it will be much, much slower as it will have to process all the non-brain voxels as well.

\section{mcDESPOT}

The \texttt{mcdespot} program is complicated, because mcDESPOT is a complicated technique. I have tried to make it as user friendly as possible. I particularly thank Anna Combes and Ruth O'Gorman for feedback and finding bugs. I suggest reading some of Sean's papers about the technique listed in section~\ref{credit} before you try processing the data. In its simplest terms the \texttt{mcdespot} technique tries to partition the signal measured in the scans into a signal coming from ``myelin water'', ``cell'' water and an optional component for CSF. These components have short, long, and very long T1s\&T2s respectively.

\texttt{mcdespot} can use any number and combination of SPGR and SSFP files as input. For output it will produce a different number of files depending on which particular model you decide to use. These will be the T1 and T2 of each component, the fraction of the PD (water) contained in them, the off-resonance and an extra parameter describing how long water stays in the myelin water component. To avoid long filenames, early on while writing \texttt{mcdespot} I decided to simply label the two (or three) components \texttt{a}, \texttt{b} and \texttt{c}. With hindsight I think this was a sensible decision, as there is some debate in the literature about what exactly should be contained in each component. Table~\ref{mcdespotfilenames} is provided to remind you what each one is. Filenames will be prefixed with \texttt{2C} when using the two component model and \texttt{3C} when using the three component model. The fraction of water contained in the intra-/extra-cellular water fraction (IEF) is not output because it is not a free parameter - if you want to see this fraction then subtract the MWF (and CSFF) from 1 using \texttt{fslmaths}, using your brain mask to represent 1.

\begin{table}\begin{center}
\begin{tabular}{c c c}
\hline
Suffix	& Unit	& Description\\
\hline
\texttt{T1\_a}	& seconds & T1 of MWF \\
\texttt{T2\_a}	& seconds & T2 of MWF \\
\texttt{T1\_b}	& seconds & T1 of IEF \\
\texttt{T2\_b}	& seconds & T2 of IEF \\
\texttt{T1\_c}	& seconds & T1 of CSF (Only for 3C model) \\
\texttt{T2\_c}	& seconds & T2 of CSF (Only for 3C model) \\
\texttt{tau\_a}	& seconds & Myelin water residence time \\
\texttt{f\_a}	& None (fraction) & MWF \\
\texttt{f\_c}	& None (fraction) & CSF Fraction (Only for 3C model) \\
\texttt{f0}		& Hertz   & Off-resonance \\
\texttt{SoS}	& None    & Sum-of-Squares of Residual \\
\hline
\end{tabular}
\caption{Filenames created by \texttt{mcdespot} and their meanings}
\label{mcdespotfilenames}
\end{center}\end{table}	

There are a lot of input options for \texttt{mcdespot}. Mostly these are for my own testing purposes. Unlike the other programs, you do not specify the input files on the command line. Instead the program asks you what type of file you wish to open, then for a filename, and then for relative scan settings. A typical use of the program (with made up flip-angles and TRs) might look like this:

\begin{lstlisting}[language=sh]
$mcdespot -v -m mask.nii.gz -o test\_
DESPOT Tools Beta 1 (Oct 14 2013)
Written by tobias.wood@kcl.ac.uk. 
Acknowledgements greatfully received, grant discussions welcome.
Reading mask file mask.nii.gz
Output prefix will be: test\_
Specify next image type (SPGR/SSFP): SPGR
Enter image path: spgr.nii.gz
Opened: spgr.nii.gz
Enter TR (seconds): 0.008
Enter 8 Flip-angles (degrees): 1 2 3 4 5 6 7 8
Specify next image type (SPGR/SSFP, END to finish input): SSFP
Enter image path: ssfp.nii.gz
Opened: ssfp.nii.gz
Enter number of phase-cycling patterns: 2
Enter 1 phase-cycles (degrees): 180 0
Enter TR (seconds): 0.005
Enter 8 Flip-angles (degrees): 1 2 3 4 5 6 7 8
Specify next image type (SPGR/SSFP, END to finish input): END
\end{lstlisting}

The program will then start processing. \texttt{mcdespot} is a slow program. Originally it took days to process a full dataset. If you use the machines on the Neuro-Analysis Network, currently it will take 6 hours to process a brain. The use of a mask is absolutely essential.

As with \texttt{despot2fm}, if you collected data on a non-3T magnet then you will have to use the \texttt{-t} option to use sensible fitting ranges for each parameter. Currently default ranges are only given for 3T and 7T data. If you use any other field strength then you need to use the \texttt{-t u} option. The program will then prompt you to enter a fitting range for each parameter in turn.

\section{Scripting}\label{scripting}

Most people will want to write a script to process a complete dataset, so that the same processing can be applied to multiple subjects. The programs listed above are designed with scripting in mind, but you will need to know about Unix ``redirection'' in order to script the entry of flip angles, TRs and other scan settings. A comprehensive, but technical, guide can be found at \url{http://www.tldp.org/LDP/abs/html/io-redirection.html}.

To use this feature, you need to create a text file that contains exactly what you would type when the program prompts you, including returns/enters. A template file (\texttt{template\_mcd\_input.txt}) which corresponds to the above use of \texttt{mcdespot} should have been provided with this manual. To specify that the program should use this file for input instead of the keyboard you use a ``\texttt{<}'' character, like so:

\begin{lstlisting}[language=sh]
$mcdespot -v -m mask.nii.gz -o test\_ <template\_mcd\_input.txt
\end{lstlisting}

You can hence have one file containing your scan settings and use it with your entire dataset. This is how I usually run the programs. If you wish to be more advanced, then look up ``HEREDOC'' on-line. This allows you to use redirection within a script, but because it is complicated to use correctly I cannot provide help.

If you do script the programs, the \texttt{-n} option suppresses the text prompts, which will make the output more intelligible.

\end{document}